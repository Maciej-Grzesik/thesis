\chapter{Wnioski}
\section{Analiza zastosowania transformerów}
Ostatecznnie zastosowany model nie odniósł sukcesu w rozpoznawaniu znaków języka migowego.
Ciężko jednoznacznie określić przyczynę takiego stanu rzeczy, jednakże można wskazać na kilka potencjalnych przyczyn.
Pierwszą z nich jest zastosowanie modelu wytrenowanego na niewystarczająco dużym zbiorze danych.
Wytrenowanie modelu na większym zbiorze danych mogłoby pozwolić na uzyskanie lepszych wyników.

\section{Perspektywa rozwoju}
Istotnym czynnikiem wymagającym dalszego rozwozju jest zastosowanie większego zbioru danych do wytrenowania modelu.
W tym celu należałoby zebrać odpowiednie dane w postaci filmów wideo, a następnie odpowiednio je przetworzyć w celu uzyskania danych w postaci sekwencji punktów charakterystycznych dłoni, rąk oraz głowy.
Obiecującym zbiorem danych jest zbiór How2Sign, który jest dostępny na licencji niekomercyjnej, jednak tym samym zawiera znaczną ilość danych.

Warto również zaznaczyć, iż zbieranie dodatkowych danych mogłoby być przeprowadzane z poziomu aplikacji mobilnej, gdzie użytkownik mógłby nagrywać swoje znaki języka migowego, a następnie przesyłać je do serwera w celu dalszej analizy.
W takim przypadku zastosowanie bazy danych przechowującej nagrane znaki języka migowego byłoby konieczne.
Takie rozwiązanie pozwoliłoby na zwiększenie ilości danych treningowych, nie tylko w angielskim języku migoweym ale i również w innych językach.

Takie rozwiązania wymagały by zaprzęgnięcia znacznej ilości zasobów obliczeniowych.
Przydatnymi narzędziami w takim przypadku zdecydowanie były rozwiązania typu Cloud Computing, które nie tylko pozwoliłyby na przechowywanie danych, ale również na ich przetwarzanie.

% Rozdział 2: Wnioski
% Rozdział „Wnioski” powinien zawierać podsumowanie tego, co udało się osiągnąć podczas pracy, wnioski płynące z przeprowadzonych badań oraz praktyczne aspekty wdrożenia rozwiązania.

% 2.1 Ocena skuteczności aplikacji
% Podsumuj, jak skuteczna jest aplikacja w rozpoznawaniu znaków języka migowego. Czy spełnia założone cele, takie jak dokładność modelu, czas odpowiedzi i łatwość użytkowania?

% Przykład:
% „Aplikacja okazała się skuteczna w rozpoznawaniu znaków języka migowego, osiągając wysoką dokładność, stabilność i płynność działania, co czyni ją użytecznym narzędziem w kontekście nauki i komunikacji.”

% 2.2 Analiza zastosowania transformatorów
% Omów, jakie korzyści płyną z użycia transformatorów w tym kontekście. Porównaj je z innymi rozwiązaniami, które mogłyby zostać zastosowane, i uzasadnij wybór transformatora.

% Przykład:
% „Wykorzystanie modelu opartego na transformatorze umożliwiło uzyskanie wysokiej precyzji w rozpoznawaniu znaków, a także pozwoliło na efektywne przetwarzanie sekwencji wideo, co w przypadku innych metod, takich jak sieci CNN, było bardziej czasochłonne.”

% 2.3 Rekomendacje na przyszłość
% Jeśli to możliwe, zaproponuj możliwe kierunki rozwoju aplikacji. Na przykład poprawa dokładności, dodanie nowych funkcji, lepsza optymalizacja dla urządzeń mobilnych czy wsparcie dla większych zbiorów danych.

% Przykład:
% „Aplikacja może zostać rozwinięta poprzez rozszerzenie bazy danych o dodatkowe znaki języka migowego, co poprawiłoby jej dokładność. Kolejnym krokiem mogłoby być zwiększenie efektywności modelu poprzez techniki takie jak pruned transformers, co zmniejszyłoby czas odpowiedzi aplikacji.”
