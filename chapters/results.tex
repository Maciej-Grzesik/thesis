\chapter{Wyniki}
\section{Funkcjonalność aplikacji}
Aplikacja mobilna została oparta o nastepujące funkcjonalności:
\begin{itemize}
    \item Rozpoznawanie znaków języka migowego: aplikacja mobilna pozwala użytkownikowi na nagranie znaku języka migowego, następnie wynik klasyfikacji przeprowadzony od strony serwera jest wyświetlany na interfejsie użytkownika.
    \item System rejestracji i logowania: aplikacja mobilna pozwala użytkownikowi na zarejestrowanie się oraz zalogowanie do systemu, zaimplementowaniu zewnętrznego serwisu FirebaseAuth.
    \item System zarządzania kontem: aplikacja mobilna pozwala użytkownikowi na zarządzanie swoim kontem, w tym zmianę hasła, wylogowanie oraz usunięcie konta. Dodatkowo użytkownik posiada możliwość zresetowania hasła w przypadku jego zapomnienia.
    \item Możliwość zmiany języka interfejsu: aplikacja mobilna pozwala użytkownikowi na zmianę języka interfejsu, dostępne są dwa języki: polski oraz angielski.
    \item Możliwość zmiany motywu interfejsu: aplikacja mobilna pozwala użytkownikowi na zmianę motywu interfejsu, dostępne są dwa motywy: jasny oraz ciemny.
\end{itemize}

\section{Testy aplikacji}
Istotnym elementem tworzenia oprogramowania jest pokrycie kodu testemi, które pozwala zweryfikować poprawność działania aplikacji oraz zapewnić jej niezadowne działanie w różnych warunkach.
Aplikacja mobilna została przetestowana przy pomocy testów jednostkowych.
Kluczową role odegrało odpowiednie podejście to architektury systemu, które pozwoliło testować poszczególne komponenty niezależnie od siebie.

\subsection{Wyniki testów}

Wszystkie kluczowe komponenty aplikacji zostały pokryte testami jednostkowymi.
Testy wykazały, że logika biznesowa zaimplementowana w BLoC działa poprawnie, a przepływ danych między warstwami w zastosowanej architekturze jest zgodny z oczekiwaniami.

\section{Interfejs użytkownika}
Podczas tworzenia oprogramowania starano się zapewnić odpowiednią strukturę interfejsu użytkownika, tak aby był on intuicyjny oraz łatwy w obsłudze.
Kluczowym aspektem podczas prototypowania wyglądu interfejsu było zapewnienie odpowiedniej palety kolorów oraz zapewnienie odpowiedniego kontrastu między poszczególnymi elementami.
Zapewnienie odowiedniej palety kolorów pozwala na zwiększenie czytelności interfejsu poprzez m.in. wyszczególnienie ważnych pod względem funkcjonalności elementów.

Zadbano również o to aby interfejs był responsywny oraz dostosowany do różnych rozdzielczości ekranów.

\section{Problemy napotkane podczas implementacji}
Największym problemem napotkanym podczas implementacji aplikacji był sposób w jaki odbywa się klasyfikacja znaków poprzez model, konkretnie, nie jest on w stanie bez odowiedniego przygotowania danych rozpoznawać więcej niż jednego znaku na raz.
W tym celu zdecydowano się na ograniczenie nagrywanego znaku do jednego poprzez ograniczenie czasu nagrywania do 2 sekund.

Nie został rozwiązany problem dużej ilości klas, a co za tym idzie, dużą ilość możliwych znaków, przez co model nie jest w stanie rozpoznawać znaków w warunkach rzeczywistych podczas użytkownia aplikacji.