\chapter{Wprowadzenie}\label{ch:intro}

\section{Motywacja}

Wykluczenie społeczne osób głuchoniemych stanowi poważny problem w społeczeństwie, wynikający z braku mozliwości komunikacji werbalnej w sferze publicznej. 
Uszkodzenie słuchu, szczególnie w życiu codziennym, powoduje trudności w~komunikacji z otoczeniem, które opiera się na konwencjonalnej mowie.
Rodzi to problem z mozliwością partycypacji w prostych aktywnościach poczynając od uczestniczenia w zyciu społecznym do m.in. jakości otrzymywanych usług medycznych.

Nalezy zaznaczyć, iz znajomość języka migowego wsród ludzi zdrowych ogranicza się do sytuacji w których osoby im bliskie są dotknięte problemem uszkodzonego słuchu.  
Język ten, będący podstawową formą komunikacji dla osób głuchoniemych nie jest szeroko nauczany a jego znajomość często wymaga zaangazowania się w dodatkowe kursy tudziez szkolenia, nierzadko płatne.

Na podstawie badań Pauliny Malczewskiej wynika, że aż 90\% ankietowanych głuchoniemych doświadcza alienacji oraz dyskryminacji ze strony słyszących \cite{malczewska2011}.
Zjawiska te niezaprzeczalnie przyczyniają się do przewlekłego obnizenia nastroju, lęku przed byciem postrzeganym przez społeczeństwo oraz ogólnym spadkiem poczucia własnej wartości i samooceny. 
Te aspekty stawiają solidny grunt pod rozwój chorób psychicznych tj. depresja, epizody nastroju depresyjnego, zaburzenia adaptacyjne czy fobii społecznej co potwierdzają poszczególne wytyczne diagnostyczne zawarte w DSM-5 \cite{american2013diagnostic}.

Istotnym jest zaadresowanie tego problemu za pomocą aplikacji wspierającej osoby głuchonieme.
Nie tylko ułatwi to komunikację, ale także przyczyni się do zwiększenia inkluzywności społecznej, zapewniając równe szanse i redukując poziom dyskryminacji.


\section{Cel}\label{sec:aim}

Głównym celem pracy jest stworzenie algorytmu opertego na transofrmerach czyli na architekturze głębokiego uczenia maszynowego, którego zadaniem będzie klasyfikacja znaków języka migowego.
Istotnym jest zasięgnięcie do rozwiązań opartych na uczeniu maszynowym gdyz rozwiązania oparte na algorytmice nie sprawdzają się w przypadkach dotyczących widzenia komputerowego, które jest znaczącym elementem tej pracy. [elaborate on this]
Umozliwi to tłumaczenie języka migowego na tekst w czasie rzeczywistym, co ułatwi komunikację osobom głuchoniemym. 

Sam algorytm zaimplementowany zostanie w aplikacji mobilnej docelowo dedykowanej na smartfony z systemem oparacyjnym Android oraz iOS. Mozliwe jest to dzięki wykorzystaniu środowiska Flutter, które pozwala na tworzenie oprogramowania opartego o język Dart, a następnie na kompilowanie kodu przy wykorzystaniu natywnych narzędzi kompilacyjnych (Android Studio dla systemu Android; Xcode dla systemu iOS).

\section{Zakres}\label{sec:scope}

W ramach pracy inżynierskiej zaprojektowany zostanie przyjazny interfejs użytkownika, który umożliwi proste korzystanie z aplikacji zarówno przez osoby głuche, jak i słyszące.
Wykorzystanie środowiska Flutter pozwoli na implementacje interaktywnego interfejsu użytkownika przy wykorzystaniu wbudowanych funkcjonalności oraz zewnętrznych modułów utrzymywanych przez społeczność programistów.
Dodatkowo zostanie wyeliminowana potrzeba pisania kodu w natywnym dla danego środowiska języku, co zapewnia aplikacji możliwość działania na różnych mobilnych systemach operacyjnych.

Zaimplementowany zostanie również model mający na celu klasyfikowanie odpowiednich filmów wideo do odpowiadającego im tesktu.
W tym etapie projektu zostaną zastosowane odpowiednie algorytmy, które umożliwią skuteczne uczenie modelu na podstawie zebranych danych dotyczących gestów języka migowego.
Następnie za pomocą odpowiednio napisanego potoku dane zostaną przesłane do modelu tym samym zaczynając proces uczenia.
Proces ten będzie obejmował zarówno fazę wstępną, w której model będzie dostosowywany do charakterystyki danych, jak i fazę walidacji, w której oceni się jego dokładność i zdolność do generalizacji gestów języka migowego.
Istotnym jest również przeprowadzenie testów funkcjonalności modelu w warunkach rzeczywistych aby potwierdzić jego poprawne działanie lub wprowadzanie ewentualnych poprawek.  
Tak skonstruowany i wyuczony model następnie zostanie zkonwertowany do odpowiedniego formatu który następnie zostanie zintegrowany z aplikacja mobilną.